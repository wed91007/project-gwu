\documentclass{article}

\usepackage{graphicx} % Required for the inclusion of images
%\usepackage{natbib} % Required to change bibliography style to APA
\usepackage{amsmath} % Required for some math elements 
\usepackage{float}
\usepackage{url}
\setlength\parindent{0pt} % Removes all indentation from paragraphs

\renewcommand{\labelenumi}{\alph{enumi}.} % Make numbering in the enumerate environment by letter rather than number (e.g. section 6)

%\usepackage{times} % Uncomment to use the Times New Roman font

%----------------------------------------------------------------------------------------
%	DOCUMENT INFORMATION
%----------------------------------------------------------------------------------------

\title{Project 2 Report} % Title

\author{Xiangyang \textsc{Han}} % Author name

\date{\today} % Date for the report

\begin{document}

\maketitle % Insert the title, author and date


% If you wish to include an abstract, uncomment the lines below
% \begin{abstract}
% Abstract text
% \end{abstract}

%----------------------------------------------------------------------------------------
%	SECTION 1 Overview
%----------------------------------------------------------------------------------------
\section{Project Overview}


\begin{tabular}{ll}
Name:&Xiangyang Han\\
Student ID:&G29562597\\
Major Resource:&ide:Spyder; Reference:wiki,stackoverflow\\
Programming Language:&python 2.7\\
Github url:&\url{https://github.com/wed91007/project-gwu}
\end{tabular}


% If you have more than one objective, uncomment the below:
%\begin{description}
%\item[First Objective] \hfill \\
%Objective 1 text
%\item[Second Objective] \hfill \\
%Objective 2 text
%\end{description}


 
%----------------------------------------------------------------------------------------
%	SECTION 2 Question 1
%----------------------------------------------------------------------------------------

\section{Parameter Estimation}

\subsection{Problem 1}
Poisson distribution has pmf as: \\
\begin{displaymath}
P(X|\lambda)=\frac{{\lambda^X}{e^{-\lambda}}}{X!}
\end{displaymath}
MLE for Poisson distribution is getting $\lambda$ for $\max P(X|\lambda)$ with knowing X.\\
Since the logarithm function is monotonically increasing, we can use log to compute MLE:\\
\begin{align*}
L(\lambda)&=\log{\prod_{i=1}^n}f(X_i|\lambda)\\
			&=\sum_{i=1}^n \log(\frac{\lambda^X e^{-\lambda}}{X!})\\
			&=-n\lambda + (\sum_{i=1}^n X_i)\log(\lambda)-\log(\sum_{i=1}^n X!)
\end{align*}\\

When derivative of $L(\lambda)=0$, it is the max:\\
\begin{displaymath}	
\frac{d}{d\lambda}L(\lambda)=0 \Leftrightarrow -n+(\sum_{i=1}^n X_i)\frac{1}{\lambda}
\end{displaymath}
Solving the equation:\\
\begin{displaymath}
\lambda_{MLE}=\hat{\lambda}=\frac{1}{n}\sum_{i=1}^n X_i
\end{displaymath}



 



\subsection{Problem 2}
Gamma distribution has pdf:\\
\begin{displaymath}
p(\lambda|\alpha,)
\end{displaymath}



%----------------------------------------------------------------------------------------
%	SECTION 3 Question 2
%----------------------------------------------------------------------------------------

\section{Decision Trees}
\subsection{H(Y)}
The total count is $3+4+4+1+0+1+3+5=21$. \\
$P(Y=+)=\frac{12}{21}=0.57$,$P(Y=-)=\frac{9}{21}=0.43$.\\
Entropy for Y is:\\
\begin{align*}
H(Y)&=-\sum P(Y)\log_2 P(Y)\\
	 &=-P(Y=+)\log_2 P(Y=+)-P(Y=-)\log_2 P(Y=-)\\
	 &=-0.57*-0.81-0.43*-1.2\\
	 &=0.98
\end{align*}

\subsection{Information gain}
In the training data: $P(X_1=T)=\frac{8}{21}=0.38$,$P(X_1=F)=\frac{13}{21}=0.62$.\\
Computing the information gain $IG(X_1)$:\\
\begin{align*}
IG(X_1)&=H(Y)-H(Y|X_1)\\
		&=H(Y)-P(X_1=T)H(Y|X_1=T)-P(X_2=F)H(Y|X_1=F)\\
		&=0.98-0.38({\frac{7}{8}\log_2 \frac{7}{8}}-{\frac{1}{8}\log_2 \frac{1}{8}})-0.62({\frac{5}{13}\log_2 \frac{5}{13}}-{\frac{8}{13}\log_2 \frac{8}{13}})\\
		&=0.98-0.38(0.17+0.38)-0.62(0.43+0.53)\\
		&=0.98-0.21-0.60
		&=0.17
\end{align*}
For $X_2$: $P(X_2=T)=\frac{}{21}$


%----------------------------------------------------------------------------------------
%	SECTION 4 Question 3
%----------------------------------------------------------------------------------------

\section{Perceptron}
\subsection{OR Function}
\subsection{XOR Function}


%----------------------------------------------------------------------------------------
%	SECTION 5 Question 4
%----------------------------------------------------------------------------------------

\section{Support Vector Machine}
\subsection{Dataset Details}
This project uses Breast Cancer Wisconsin (Diagnostic) Data Set. Features are computed from a digitized image of a fine needle aspirate (FNA) of a breast mass. They describe characteristics of the cell nuclei present in the image. 
The attributes contains ID, Diagnosis(malignant or benign) and ten real-valued features for each cell nucleus.
\subsection{Algorithm Description}
\subsection{Algorithm Results}
\subsection{Runtime}







%----------------------------------------------------------------------------------------
%	SECTION 6
%----------------------------------------------------------------------------------------

%\section{Answers to Definitions}



%----------------------------------------------------------------------------------------
%	BIBLIOGRAPHY
%----------------------------------------------------------------------------------------

%\bibliographystyle{apalike}

%\bibliography{}

%----------------------------------------------------------------------------------------


\end{document}